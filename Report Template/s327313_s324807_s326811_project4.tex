\documentclass[10pt,twocolumn,letterpaper]{article}
\usepackage{graphicx}
\usepackage{amsmath}
\usepackage{amssymb}
\usepackage{booktabs}
\usepackage{float}
\usepackage[pagebackref,breaklinks,colorlinks]{hyperref}
\usepackage[capitalize]{cleveref}
\usepackage{abstract} % Add this package for abstract formatting
\setlength{\topmargin}{-0.7in}
\setlength{\textheight}{9.5in} 
\setlength{\columnsep}{20pt}
\crefname{section}{Sec.}{Secs.}
\Crefname{section}{Section}{Sections}
\Crefname{table}{Table}{Tables}
\crefname{table}{Tab.}{Tabs.}

\begin{document}

\title{Real-time Domain Adaptation in Semantic Segmentation}

\author{Ianniello Luca, 
Martone Raffaele,
Sirica Antonio\\}


%%%%%%%%% ABSTRACT
\twocolumn[
\maketitle
\begin{@twocolumnfalse}
\begin{abstract}
\textit{ 
\small
This study addresses the problem of real-time domain adaptation in semantic segmentation, focusing on bridging performance gaps between source and target domains. Leveraging the LoveDA dataset, which encompasses rural and urban domains, we evaluate the performance of state-of-the-art models, including DeepLabv2 and PIDNet, under domain adaptation settings. We explore a range of adaptation techniques, such as data augmentation, adversarial learning, domain adaptation via cross-domain mixed sampling (DACS), Prototype-based Efficient MaskFormer (PEM), and unpaired image-to-image translation with CycleGAN. Our findings reveal the strengths and limitations of these methods in improving segmentation performance across domains while preserving computational efficiency.
}
\end{abstract}
\vspace{1.2em}
\end{@twocolumnfalse}
]


%%%%%%%%% BODY TEXT
\section{Introduction}
\label{sec:intro}

Semantic segmentation is a fundamental task in computer vision that involves partitioning an image into semantically meaningful regions, typically corresponding to different object classes. This task is crucial for various applications, including autonomous driving, medical imaging, and remote sensing. Recent advancements in deep learning have significantly improved the performance of semantic segmentation models \cite{hao2020brief}. 

While deep learning models such as DeepLab \cite{chen2017deeplab} and PIDNet \cite{feng2021pidnet} have achieved remarkable performance on benchmark datasets, their generalization to unseen domains remains a significant obstacle. Domain shifts, arising from differences in image characteristics, environments, and data distributions, often result in substantial performance degradation.

Domain adaptation aims to bridge the performance gap between source and target domains, enabling models trained on one domain to generalize effectively to another. The LoveDA dataset \cite{wang2021loveda}, with its distinct rural and urban domain settings, provides a robust benchmark for evaluating domain adaptation techniques in semantic segmentation. 

This project investigates real-time domain adaptation techniques for semantic segmentation, focusing on the LoveDA dataset. We evaluate state-of-the-art models, such as DeepLabv2 and PIDNet, under various domain adaptation scenarios. Specifically, we analyze the challenges of adapting models from rural to urban domains and vice versa, and we explore solutions including data augmentation, adversarial training \cite{tsai2018advlearning}, image-to-image translation approaches (DACS) \cite{tranheden2021dacs}, real-time networks (PEM) \cite{cavagnero2024pem}, and style transfer preprocessing models, like CycleGAN \cite{zhu2020cyclegan}. By experimenting with these approaches, we aim to identify effective methods for improving domain adaptation performance while maintaining real-time capabilities.

\section{Related Work}

\subsection{DeepLabv2}

\subsection{PIDNet}

\subsection{LoveDa Dataset}

\section{Methodology}

\subsection{DeepLab2 implementation}

\subsection{PIDNet implementation}

\subsection{Data Augmentation}

\subsection{Adversarial Learning}

\subsection{Domain Adaptation via Cross-domain Mixed Sampling (DACS)}

\subsection{Prototype-based Efficient MaskFormer (PEM)}

\subsection{Unpaired Image-to-Image Translation with CycleGAN}

\section{Experiments and results}

\subsection{DeepLabv2 Experiments}

\subsection{PIDNet Experiments}

\subsection{Data Augmentation Experiments}

\subsection{Adversarial Learning Experiments}

\subsection{DACS Experiments}

\subsection{PEM Experiments}

\subsection{CycleGAN Experiments}

\section{Conclusion}
%%%%%%%%% REFERENCES

\bibliographystyle{plain}
\bibliography{references}

\end{document}